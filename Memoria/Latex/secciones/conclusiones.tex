\newgeometry{margin=85pt}
\chapter*{Conclusiones y líneas futuras}
\setcounter{chapter}{7} 
\setcounter{section}{0}
\addcontentsline{toc}{chapter}{Conclusiones y líneas futuras}

\section{Conclusiones}
El mundo de los Sistemas de información Geográfica es muy extenso y se relaciona con otros muchos campos. 
En este trabajo hemos visto una pequeña parte de su potencial para el análisis espacial y 
algunos de los modelos más recientes de aprendizaje automático y aprendizaje profundo. 
Tanto GIS como la AI son campos en proceso de expansión y su intersección, la GeoAI, resulta necesaria para analizar el gran volumen de datos espaciales que existen en la actualidad. 
En las primeras fases de comercialización de los GIS se daba más importancia a la creación de un software robusto y que abarcara la resolución de la mayoría de los problemas.
En estos últimos años, los esfuerzos se han centrado en el tratamiento de los grandes volúmenes de datos espaciales que se generan y por ello actualmente los GIS 
tienen una estrecha relación con campos como la Ciencias de Datos.
Los lenguajes de programación, como R o Python, ofrecen un entorno de programación espacial que resulta más cómodo para la manipulación y análisis de datos espaciales.
Las herramientas GIS, como QGIS o ArcGIS, permiten realizar prácticamente las mismas operaciones sobre los datos, pero aportan más valor que los entornos de programación 
a la hora de filtrar, seleccionar y visualizar datos espaciales. 

\section{Líneas futuras}
Las tecnologías que han sido objeto de investigación en este trabajo permiten continuar con su investigación tomando distintos caminos.
Algunos de ellos son:
\begin{itemize}
    \item Conocer nuevas fuentes de datos espaciales y gestionar la información empleando distintas bases de datos espaciales.
    \item Conocer y aplicar más métodos de análisis espacial.
    \item Conocer e implementar más modelos de Inteligencia Artificial aplicados en el ámbito geoespacial.
    \item Adentrar en el mundo de la Teledetección y emplear algoritmos de superresolución a imágenes satelitales, con el fin de obtener mejores resultados en modelos de aprendizaje profundo. 
\end{itemize} 

