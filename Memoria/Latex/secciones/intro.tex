\newgeometry{margin=85pt}
\chapter*{Introducción y objetivos}
\setcounter{chapter}{1} 
\setcounter{section}{0}
\addcontentsline{toc}{chapter}{Introducción y objetivos}
\section{Introducción}
Este trabajo se centra en el ámbito de la Inteligencia Artificial (IA), aplicado en los Sistemas de Información Geográfica (SIG).
El objetivo es buscar la forma en la que la IA ayuda a mejorar el análisis de la información geográfica, favoreciendo así la toma de decisiones.

En estos últimos años ha habido importantes avances en el campo de la IA, ayudando en tareas como el reconocimiento de imágenes o la traducción de textos.
La intersección de los SIG y la IA está creando enormes oportunidades que antes no eran posibles, por ejemplo,
a aumentar el rendimiento de los cultivos a través de la agricultura de precisión,
comprender los patrones delictivos y predecir cuándo llegará la próxima gran tormenta y estar mejor equipados para manejarla.

\section{Objetivos}

\begin{itemize}
  \item Comprender las capacidades de los Sistemas de Información Geográfica y las formas de representación de los datos geográficos.
  \item Conocer los distintos tipos de Sistemas de Referencia de Coordenadas y las distintas formas de representación de la superficie terrestre.
  \item Conocer y entender modelos estadísticos para el análisis de datos geográficos.
  \item Conocer y entender modelos de IA aplicados en el ámbito geoespacial.
  \item Realizar una prueba de concepto con uno de los modelos aplicado a un caso real.
\end{itemize}

\section*{Palabras clave}

\begin{itemize}
  \item Información geográfica/geoespacial/espacial.
  \item Propiedades descriptivas/atributos.
  \item Capas/datos vectoriales.
  \item Capas/datos ráster.
  % \item Localización/ubicación/posición.
  \item Modelo/clasificador.
  \item Características/propiedades/atributos.
  \item Clases/categorías/etiquetas.
\end{itemize}

\section*{Símbolos y abreviaciones}

\begin{itemize}
  \item «GIS» Geographical Information System.
  \item «CRS» Coordinate Reference System.
  \item «GeoAI» Geospatial Artificial Intelligence.
  \item «INSPIRE» Infraestructure for Spatial Information in Europe
  \item «ESRI» Environmental Systems Research Institute
  \item «OGC» Open Geospatial Consortium
  \item «WMS» Web Map Service
  \item «WFS» Web Feature Service
  \item «IDENA» Infraestructura de Datos Espaciales de Navarra
\end{itemize}




